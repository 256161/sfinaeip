Реализовать функцию печати условного IP-\/адреса. Условность его заключается в том, что количество элементов не обязательно должно быть равно 4-\/ ём или 8-\/ми, а также каждый элемент не обязательно должен быть числом из диапазона 0..255. От идеи IP-\/адреса остаётся фактически только вывод элементов через {\ttfamily .} (символ точки). Функцию нужно реализовать для различных входных параметров путём использования механизма SFINAE. Всего нужно выполнить 3 обязательных и один опциональный вариант функции.
\begin{DoxyEnumerate}
\item Адрес может быть представлен в виде произвольного целочисленного типа. Выводить побайтово в беззнаковом виде, начиная со старшего байта, с символом {\ttfamily .} (символ точки) в качестве разделителя. Выводятся все байты числа.
\item Адрес может быть представлен в виде строки. Выводится как есть, вне зависимости от содержимого.
\item Адрес может быть представлен в виде контейнеров {\ttfamily std\+::list}, {\ttfamily std\+::vector}. Выводится полное содержимое контейнера поэлементно и разделяется {\ttfamily .} (символом точка). Элементы выводятся как есть.
\item Опционально адрес может быть представлен в виде {\ttfamily std\+::tuple} при условии, что все типы одинаковы. Выводится полное содержимое поэлементно и разделяется {\ttfamily .} (одним символом точка). Элементы выводятся как есть. В случае, если типы кортежа не одинаковы, должна быть выдана ошибка при компиляции кода. Прикладной код должен содержать следующие вызовы\+: print\+\_\+ip( int8\+\_\+t\{-\/1\} ); // 255 print\+\_\+ip( int16\+\_\+t\{0\} ); // 0.\+0 print\+\_\+ip( int32\+\_\+t\{2130706433\} ); // 127.\+0.\+0.\+1 print\+\_\+ip( int64\+\_\+t\{8875824491850138409\} );// 123.\+45.\+67.\+89.\+101.\+112.\+131.\+41 print\+\_\+ip( std\+::string\{“\+Hello, World!”\} ); // Hello, World! print\+\_\+ip( std\+::vector$<$int$>$\{100, 200, 300, 400\} ); // 100.\+200.\+300.\+400 print\+\_\+ip( std\+::list$<$shot$>$\{400, 300, 200, 100\} ); // 400.\+300.\+200.\+100 print\+\_\+ip( std\+::make\+\_\+tuple(123, 456, 789, 0) ); // 123.\+456.\+789.\+0 
\end{DoxyEnumerate}